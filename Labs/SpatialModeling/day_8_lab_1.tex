\documentclass[12pt]{article}
\usepackage{amssymb,tabularx,multirow,amsmath,natbib, epsfig, threeparttable, amstext, subfigure}
\usepackage[left=.5in,top=.75in,right=.5in,nohead,nofoot]{geometry}
\setlength{\parskip}{12pt}

\title{Day 8, PM Lab \\ Intro to Spatial Statistics \\ Bayesian Geostatistics}
\date{}
\begin{document}
\maketitle

\section*{Overview}

In this lab, we are going to 1.) simulate geostatistical data, 2.) examine spatial dependence in errors, 3.) fit a Bayesian geostatistical model to simulated data using a.) JAGS and b.) spBayes, 4.) obtain spatial predictions (using spBayes), 5.) compare spatial predictions to ``true'' simulated process, 6.) create Bayesian model for Iowa temperature data and obtain predictions in a rectangular region encompassing the data. 

\section*{Objectives}
The objectives of this lab are to:
\begin{enumerate}
  \item Gain familiarity with spatial processes and models.
  \item Obtain experience simulating Gaussian spatial processes.  
  \item Learn how to manipulate spatial data in R.  
  \item Use variograms to assess spatial structure in data and/or residuals.  
  \item See how to fit Bayesian geostatistical models and obtain spatial predictions.
\end{enumerate}

\section*{Procedure}
\begin{enumerate}
  \item Open the MCMC algorithm file called `lab\_8\_script.R' in a text editor (or R).  Examine the different sections of this script which are denoted by the comments `\#\#\#\#'.  
  \item In R, change directory to the lab folder. 
  \item In this lab, we are first going to   
  \item Following the R script:
  \begin{enumerate}
    \item Load packages:  geoR, maps, mvtnorm, rjags, spBayes, rgdal, and maptools.  Download them first if you don't have them installed.
    \item Simulate data based on a continuous spatial process encompassing the state of Utah:
    \begin{enumerate}
      \item Convert Utah map to UTM coordinates. 
      \item Find the boundaries of Utah. 
      \item Specify points locations that span the spatial domain. 
      \item Make a map to check these locations.  Note that even though these are truly point locations, we are going to visualize them in a regular grid for convenience.  They are NOT areal data in this case, as a different simulation and modeling procedure would be used in that case.
      \item Create Euclidean distance matrix that specifies the distance between all points in full domain. 
      \item Specify an exponential covariance model: $\Sigma_{i,j}=\sigma^2 \exp(-d_{i,j} \phi)$.  Note that in this specification the range parameter $\phi$ is on top in the exponential (preferred by the spBayes package).   
      \item Simulate the spatially correlated error process from a multivariate normal: $\boldsymbol\varepsilon \sim \text{N}(\mathbf{0}, \boldsymbol\Sigma)$.  
      \item Specify how much actual ``data'' you want keep.  Here we take a subset (perhaps randomly) of the simulated process and act like we don't have the rest, that way we can assess our predictive ability at those ``missing'' locations.
      \item Make map of these correlated errors.
    \end{enumerate}
  \item Create an empirical semi-variogram to check the spatial structure evident in the correlated errors and compare with ``known'' simulated semi-variogram. 
  \item Create a couple of spatially-structure covariates.  Here we just make up some interesting functions to construct $\mathbf{X}$.
  \item Choose a known set of values for the regression coefficients $\boldsymbol\beta$.  
  \item Create observed data vector $\mathbf{y}=\mathbf{X}\boldsymbol\beta + \boldsymbol\varepsilon$. 
  \item Make figure containing maps of $\mathbf{y}$, $\boldsymbol\epsilon$, and the covariates.  
  \item Use JAGS to fit model:
  \begin{enumerate}
    \item Set up variables so that we can use JAGS to fit geostatistical Bayesian model.   
    \item View the JAGS model specification `lab\_8\_jags\_model.txt' file to see how this spatial model might be handled.  
    \item Fit model using JAGS.  Note that we only use 1000 iterations to get a feel for how this model is fit.  JAGS is too slow for this problem to obtain a large number of MCMC samples and it is WAY too slow to produce spatial predictions.  
    \item View MCMC trace plots based on JAGS output. 
  \end{enumerate}
  \item Use spBayes to fit model 
  \begin{enumerate}
    \item Set up variables so that we can use spBayes to fit geostatistical Bayesian model.
    \item Use `spLM' to first fit the model. 
    \item Use `spRecover' to obtain MCMC samples for $\boldsymbol\beta$ so that we can assess trace plots.
    \item Check trace plots of model parameters for convergence.  
    \item Use `spPredict' to obtain predictions and prediction standard deviations at all locations in the spatial domain.  Note that these quantities represent the mean and standard deviation of the spatial posterior predictive distribution.
    \item Make figure that shows maps of the `true' process, data, predictions, and the prediction standard deviation.   
  \end{enumerate}
  \end{enumerate}
\end{enumerate}

\section*{Challenge}
\begin{enumerate}
  \item Read in the Iowa temperature data in file `iowa\_temperature.txt'. 
  \item Create map of these data using circles with size corresponding to temperature with Iowa state boundary overlaid. 
  \item Using the spatial coordinates as covariates, create a Bayesian predictive surface map over the region encompassing the data.  Also construct a map illustrating the uncertainty in predictions.  
\end{enumerate}


\end{document}
